\documentclass[fleqn]{article}[11pt]
\usepackage{amsmath} 
\usepackage{amssymb}
\usepackage[english]{babel}
\usepackage[autostyle]{csquotes}
\usepackage[margin=0.5in]{geometry}
\newcommand{\ds}{\displaystyle}
\usepackage{graphicx}
\usepackage[oztex]{harpoon}
\usepackage{romannum}
 \usepackage{enumitem}

\begin{document}
	
\begin{center}\section*{MATH 316D W09}\end{center}
\subsection*{DD1 Individual Quiz}

\begin{enumerate}
	\item \textbf{Keep}.
	
	\item \textbf{Keep}.
	
	\item \textbf{Change} to, ``Let \(\mathbf{A}=\begin{bmatrix}
									2 & 4 \\
									1 & 2 \\
								\end{bmatrix}\); one of the eigenvalues of $\mathbf{A}$ is:
		\begin{enumerate}
			\item 6
			\item 0 $\implies$ \textbf{Correct}
			\item -2
			\item None of the above.
		\end{enumerate}
		
	 \item \textbf{Keep}.
	 
	 \item\textbf{Change} to, ``Given the definition of an eigenpair as, $\mathbf{A}\overrightharp{v}=\lambda \overrightharp{v}$, where $\mathbf{A}$ is an $n\times n$ matrix and $\overrightharp{v}$ is the non-zero eigenvector. How is the action of $\mathbf{A}$ equivalent to the scalar multiplication of $\lambda$, the eigenvalue?" 
	 
	 	\begin{enumerate}
	 		\item It is the same geometrically in that it will flip, stretch, or shrink the vector equivalently. $\implies$ \textbf{Correct}
			\item $\lambda$ is equivalent to the determinant of $\mathbf{A}$.
			\item All of the above.
			\item The scalar multiplication of $\lambda$ is not equivalent to the action of $\mathbf{A}$ but depends on the column vector $\overrightharp{v}$.
		\end{enumerate}
	 
\end{enumerate}

\subsection*{DD2 Group Quiz (\textit{Exam \Romannum{3} Review})}

\begin{enumerate}
	\item \textbf{Keep}.
	
	\item \textbf{Change} to, ``Let \(\mathbf{A}=\begin{bmatrix}
									1 & 1 & 1 \\
									0 & 1 & 1 \\
									0 & 0 & 1 \\
								\end{bmatrix}\).
		\begin{enumerate}[label=\alph*.]
			\item What is the span of the columns of $\mathbf{A}$?
			\item Does $\mathbf{A}^{-1}$ exist? If it does, find it. If it does not, explain why it does not exist.
			\item What is the volume of the parallelepiped spanned by the columns of $\mathbf{A}$?
			\item Write out the linear system represented by $\mathbf{A}\overrightharp{x}=\overrightharp{b}$, if \(\overrightharp{b}=\begin{bmatrix}	
												-1\\
												2 \\
												3 \\													\end{bmatrix}\).
												
			\item Is $\overrightharp{b}$ in the span of the columns of $\mathbf{A}$? If it is, write $\overrightharp{b}$ as a linear combination of the columns of $\mathbf{A}$.
		\end{enumerate}

	\item \textbf{Change} to, ``Mark each statement True or False. Justify your answer. Let S be a set of n vectors in $\mathbb{R}^m$."
	
		\begin{enumerate}
			\item If $n>m$ the elements of $S$ are linearly independent.
			
		\end{enumerate}
	
	\item \textbf{Keep}.
	
	\item \textbf{Keep} but please make sure that the questions reads as follows: ``Compute the eigenvalues and eigenvectors for 
	
	\(\mathbf{A}=\begin{bmatrix}
	1 & -1 \\
	-1 & -1 \\
  \end{bmatrix}\). Describe the action of $\mathbf{A}$ on a vector $\overrightharp{x}$ in $\mathbb{R}^2$."
  
	\item \textbf{Keep}.
	
	\item \textbf{Keep} but please make sure that the question reads as follows: ``Determine all values of $h$ such that the augmented system is consistent, \(\begin{bmatrix} 1 & h & 3 \\ 
					    2 & h & 6 \\
		    \end{bmatrix}\)."
		    
	\item \textbf{Keep}.
	
	\item \textbf{Keep}.
	
	\item \textbf{Keep}.
\end{enumerate}

\subsection*{\textbf{KEY} - Exam \Romannum{3} Review}

\begin{enumerate}
	\item $\lambda=1$
	
	\item
		\begin{enumerate}[label=\alph*.]
			\item \(Span(\mathbf{A})= c_1\begin{bmatrix} 1 \\ 0 \\ 0 \end{bmatrix}+c_2\begin{bmatrix} 1 \\ 1 \\ 0 \end{bmatrix}+c_3\begin{bmatrix} 1 \\ 1 \\ 1 \end{bmatrix}\)
			
			\item Yes. \(\mathbf{A}^{-1}=\begin{bmatrix}									    			1 & -1 & 0 \\
								0 & 1 & -1 \\
								0 & 0 & 1 \\
							    \end{bmatrix}\)
			
			\item $V=1$
			
			\item \begin{eqnarray*}
					x_1+x_2+x_3=-1 \\
					x_2+x_3=2 \\
					x_3=3 \\
				\end{eqnarray*}
			
			\item Yes.
		\end{enumerate}
	
	\item 
		\begin{enumerate}
			\item $\implies$ \textbf{False} 
			\item $\implies$ \textbf{True}
			\item $\implies$ \textbf{True}
			\item $\implies$ \textbf{True}
		\end{enumerate}
	
	\item 
		\begin{enumerate}[label=\alph*.]
			\item $-2$
			\item $-8$
			\item $-2$
			\item $-\frac{1}{2}$
			\item This operation is not possible because...
		\end{enumerate}
		
	\item The eigenvalues of $\mathbf{A}$ are $\lambda_1=\sqrt{2}$ and $\lambda_2=-\sqrt{2}$, with corresponding eigenvectors of $\overrightharp{x}_1=\begin{bmatrix} 
								-1-\sqrt{2} \\
								1 \\
							\end{bmatrix}$,
 and $\overrightharp{x}_2=\begin{bmatrix} 
								-1+\sqrt{2} \\
								1 \\
							\end{bmatrix}$. The action of $\mathbf{A}$ on $\overrightharp{x}$ is to stretch the vector $\overrightharp{x}$ by a factor of $\sqrt{2}$.

	
	\item \(\overrightharp{r}(t)=\begin{bmatrix}
							-\frac{5}{2} \\[3pt]
							2 \\
						 \end{bmatrix}t+
						 \begin{bmatrix}
							\frac{1}{6} \\[3pt]
							\frac{1}{3} \\
						 \end{bmatrix}\) or 
		\(\overrightharp{r}(t)=\begin{bmatrix}
							-5 \\[3pt]
							4 \\
						 \end{bmatrix}t+
						 \begin{bmatrix}
							\frac{1}{6} \\[3pt]
							\frac{1}{3} \\
						 \end{bmatrix}\)
	
	\item $\implies$ \textbf{All reals}.
	
	\item 
		\begin{enumerate}[label=\alph*.]
			\item An appropriate answer will be along the lines of: ``A pivot point is a point in a given matrix, $\mathbf{A}$, that corresponds to a leading 1 when that matrix is in RREF." An appropriate equation will be of the form: \(\begin{bmatrix}
		 	1 & 0 \\
			0 & 1 \\
		 \end{bmatrix}\).
			
			\item An appropriate answer is along the lines of: ``Two matrices are row equivalent when any number of elementary steps can be taken to transform one matrix into the other." Some elementary steps that may be mentioned are: 1) \textit{replacement}, where a row is put back after being added to or subtracted by another row times a constant, 2) \textit{interchangeability}, rows can be swapped as long as their columns do not shift position, and 3) \textit{scaling}, every entry in each row can be multiplied by a non-zero constant.
		\end{enumerate}
	
	\item An appropriate answer to this question may be, ``Yes. A $2\times3$ linear system, when augmented, can be inconsistent because..."
	
	\item An example of a matrix with complex eigenvalues is, \(\mathbf{T}=\begin{bmatrix}
			4 & -4 \\
			5 & -4 \\
		    \end{bmatrix}\). The eigenvalues and eigenvectors of this matrix are as follows: $\lambda_1=2\mathrm{i}$ and $\lambda_2=-2\mathrm{i}$, with the corresponding eigenvectors being \(\overrightharp{v}_1=\begin{bmatrix}
		    									4+2\mathrm{i} \\ 5\\													      \end{bmatrix}\) and \(\overrightharp{v}_2=\begin{bmatrix}
		    			4-2\mathrm{i} \\ 5\\												 	  \end{bmatrix}\).
	
\end{enumerate}

\end{document}
\documentclass[fleqn]{article}[11pt]
\usepackage{amsmath} 
\usepackage{amssymb}
\usepackage[utf8]{inputenc}
\usepackage[usenames, dvipsnames]{color}
\usepackage[english]{babel}
\usepackage[autostyle]{csquotes}
\usepackage[margin=0.5in]{geometry}
\newcommand{\ds}{\displaystyle}
\usepackage{graphicx}
\usepackage{harpoon}
\setlength\mathindent{15pt}
          

\begin{document}
	
\begin{center}\section*{MATH 316D W07}\end{center}
\subsection*{DD1 Individual Quiz}

\textcolor{red}{\textbf{This quiz has most of the answers currently keyed incorrectly, please make sure that these are the correct answers.}}

\begin{enumerate}
	\item \textbf{Delete}, previously question 1.
	\item \textbf{Add}. ``Given the following matrix find the solution by hand using Gaussian Elimination, and choose the answer that most accurately represents your work."
	
	 \begin{equation*}
		\begin{bmatrix} 1 & 3 & 2 & 5 \\
		0 & 1 & -4 & 1 \\
		0 & 0 & 1 & 7 \\
		\end{bmatrix}
	\end{equation*}
	
		\begin{enumerate}
			\item $\begin{bmatrix}-96 & 29 & 7\end{bmatrix}^{T}$ $\implies$ \textbf{Correct}
			\item $\begin{bmatrix}7 & \frac{3}{2} & -\frac{13}{4}\end{bmatrix}^{T}$
			\item $\begin{bmatrix}\frac{37}{5} & -\frac{8}{5} & \frac{6}{5}\end{bmatrix}^{T}$
			\item None of the above.
		\end{enumerate}
	\item \textbf{Keep}, previously question 2. $\implies$ The correct answer is \textbf{False}.
	\item \textbf{Keep}, previously question 3.  $\implies$ The correct answer is \textbf{False}.
	\item \textbf{Change}, (previously question 4) to, ``If a system has a free variable present, then the system has infinitely many solutions. Is this system linearly independent or linearly dependent?"
	
		\begin{enumerate}
			\item Dependent $\implies$ \textbf{Correct}
			\item Independent
		\end{enumerate}
	\item \textbf{Keep}, previously question 5. $\implies$ The correct answer is \textbf{False}.
	\item \textbf{Keep}, previously question 6. $\implies$ The correct answer is \textbf{False}.
	\item \textbf{Change} to, ``A consistent system is one with at least one solution."
	
		\begin{enumerate}
			\item True $\implies$ \textbf{Correct}
			\item False
		\end{enumerate}
		
	\item \textbf{Add}. ``Given the following system of linear equations, create a matrix that represents this system, and then choose the answer that best represents your work."
	
		\begin{eqnarray*}
			x_{1}-x_{3}=4 \\
			-x_{2}+5x_{3}=1 \\
			2x_{1}-4x_{2}+3x_{3}=2 \\
		\end{eqnarray*}
			\begin{enumerate}
				\item \(\begin{bmatrix}
						1 & -1 & 4 & 0 \\
					       -1 & 5 & 1 & 0 \\
						2 & -4 & 3 & 2 \\
					 \end{bmatrix}\)
				\item \(\begin{bmatrix}
						1 & 0 & -1 & 0 \\
						0 & -1 & 5 & 0 \\
						2 & -4 & 3 & 0 \\
					 \end{bmatrix}\)
			
				\item \(\begin{bmatrix}
						1 & 0 & -1 & 4 \\
						0 & -1 & 5 & 1 \\
						2 & -4 & 3 & 2 \\
					 \end{bmatrix}\) $\implies$ \textbf{Correct}

				\item None of the above.
			\end{enumerate}
	\item \textbf{Add}. ``From the previous question, is this linear system homogenous or non-homogeneous?"
	
		\begin{enumerate}
			\item Homogenous
			\item Non-homogeneous
			\item There is not enough information to tell.
		\end{enumerate}
\end{enumerate}

\subsection*{DD2 Group Quiz}

\textcolor{red}{\textbf{This quiz is currently keyed correctly.}}

\begin{enumerate}
	\item \textbf{Keep}.
	\item \textbf{Add}. ``Determine whether the given set $S$ is linearly independent or linearly dependent."
	
	\(S=\{\mathbf{v}_{1},\mathbf{v}_{2}\}\) where \(\mathbf{v}_{1}= \begin{bmatrix} 1 & 0 \end{bmatrix}^{T}\) and \(\mathbf{v}_{2}= \begin{bmatrix} 0 & 1 \end{bmatrix}^{T}\).
	
		\begin{enumerate}
			\item $S$ is linearly dependent.
			\item $S$ is linearly independent. $\implies$ \textbf{Correct}
			\item There is not enough information to tell.
		\end{enumerate}
	\item \textbf{Keep}, previously question 2.
	\item \textbf{Keep}, previously question 3.
	\item \textbf{Keep}, previously question 4. 
	\item \textbf{Delete}, previously question 5.
	\item \textbf{Keep}, previously question 6.
	\item \textbf{Keep}, previously question 7.
	\item \textbf{Keep}, previously question 8.
	\item \textbf{Keep}, previously question 9.	 
\end{enumerate}

\subsection*{DD3 Weekly Quiz}

\textcolor{red}{\textbf{This quiz is currently keyed correctly.}}

\begin{enumerate}
	\item \textbf{Keep}.
	\item \textbf{Keep}.
	\item \textbf{Keep}.
	\item \textbf{Keep}.
	\item \textbf{Keep} but please fix the format of the information; in i-Learn words don't appear in the right places. Question six's information is a good template.
	\item \textbf{Keep}.
	\item \textbf{Keep} but please fix the submission box; in i-Learn it appears below where it should.
	\item \textbf{Keep}.
	\item \textbf{Keep} and please make sure that the answer is along the lines of, ``No, $S$ cannot span $\mathbb{R}^3$, because, by virtue of \textit{Theorem 1.6.1} in order for $S$ to span $\mathbb{R}^3$ it must have a pivot position in every row, which it cannot."
	\item \textbf{Keep} and please make sure that an appropriate answer is as follows: \(\overrightharp{x}_{h}=x_{3}\begin{bmatrix}
					1 \\
					1 \\
					1 \\
				\end{bmatrix}\),
		    \(\overrightharp{x}_{p}=x_{3}\begin{bmatrix}
					1 \\
					1 \\
					1 \\
				\end{bmatrix} + \begin{bmatrix}
					2 \\
					-1 \\
					0 \\
				\end{bmatrix}\), and therefore;
	\(\overrightharp{x}_{g}=2x_{3}\begin{bmatrix}
					1 \\
					1 \\
					1 \\
				\end{bmatrix} + \begin{bmatrix}
					2 \\
					-1 \\
					0 \\
				\end{bmatrix}=
				\begin{bmatrix}
					2x_{3}+2 \\
					2x_{3}-1 \\
					2x_{3} \\
				\end{bmatrix}\). Note that the solution may be parameterized and thus $x_{3}$ can be of the form $t, s, r,$ etc.
 	
\end{enumerate}

\end{document}